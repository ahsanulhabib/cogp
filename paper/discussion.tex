We have presented scalable multioutput GPs for learning of correlated functions.
The formulation around the inducing variables and their encapsulation of sufficient statistics was shown to be conducive to not only effective but also scalable joint learning under sparsity. 
We note that although our large scale experiment was done with over 40,000 observations -- the largest publicly available multioutput dataset we could find, the model can easily handle much bigger datasets.

We discuss several interesting extensions to this work.
First, consider the model for the single GP case, i.e., $P = 1$. By using $Q > 1$ sparse processes, the GP can now have multiple set of inducing variables.
These can be make local to capture varying characteristics of the latent process in different local input regions, for e.g. similar to \citep{nguyen2014fast}, leading to scalable GPs with non-stationary properties. 

A further extension is to replace the constant mixing weights in the model with input-dependent coefficients like in the Gaussian process regression networks \citep[GPRN][]{wilson-et-al-icml-12} to accommodate adaptive dependencies among the outputs.
We leave the details of these extensions to future work.