%\subsection{Variational Inference in Sparse GPs Revisited}
%
%In this section we review inference in sparse GPs (as presented in Titsias).
%The posterior in an augmented model is $p(\f, \u | \y) = p(\f | \u, \y) p(\u | \y)$.
%The key property of such augmented GPs  is the notion of \textit{sufficient statistics}: given the inducing points $\u$, the latent values $\f$ are independent with any other set of latent values (e.g. the test set).
%In the optimal setting when $\u$ is the sufficient statistics of $\f$, it should hold that $p(\f | \u, \y) = p(\f | \u)$ as $\y$ is only the noisy version of $\f$.
%This leads to choosing a variational approximation of the posterior which factorizes as $q(\f, \u | \y) = p(\f | \u) q(\u | \y)$.
%Since the conditional $p(\f | \u)$ is known, variational inference becomes learning an optimal posterior $q(\u | \y)$ only. \\
%
%\noindent Also as a consequence of sufficient statistics, the approximate prediction for test targets $\vfstar$ at test inputs $\X_*$ is
%\begin{align}
%\nonumber
%p(\vfstar | \y, \X_*) &= \int p(\vfstar | \f, \u, \X_*) q(\f, \u | \y) \der \f \der \u \\
%\nonumber
%&= \int p(\vfstar | \u) q(\u| \y) p(\f | \u) \der \f \der \u \\ \nonumber
%&= \int p(\vfstar | \u) q(\u| \y) \der \u \\
%\label{eq:sorprediction}
%&= \Normal(\vfstar; \bs{\mu_*},\vec{s_*})
%\end{align}
%where,
%\begin{align}
%\nonumber
%\bs{\mu_*} &= \K_{*z} \K_{zz}^{-1}\m \\ 
%\nonumber
%\S_* &= \K_{**} - \K_{*z} \left(\K_{zz}^{-1} - \K_{zz}^{-1} \S \K_{zz}^{-1} \right) \K_{*z}^T.
%\end{align}
% Here $\K_{*z}$ is the covariance matrix between test and inducing inputs, $\K_{zz}$ is the covariance matrix of the inducing inputs.

Inference in this model is done with variational inference.
In section \ref{sec:variationalLowerBound}, we derive a lower bound of the marginal likelihood which has the novel property of factorizing over the data points.
The next section takes advantage of this factorization to derive stochastic variational inference which allows the model to scale to massive datasets.
Section \ref{sec:hyperparameters} discusses learning of the inducing inputs and other hyperparameters.

\subsection{VARIATIONAL LOWER BOUND \label{sec:variationalLowerBound}}
\newcommand{\ug}{\u_g}
\newcommand{\uh}{\u^h}
\newcommand{\mgj}{\m_j}
\newcommand{\mhi}{\m^h_i}
\newcommand{\Sgj}{\S_j}
\newcommand{\Shi}{\S^h_i}
Our inference task is to find the posterior $p(\g, \h, \u, \v | \y)$, for which we use variational inference \citep{jordan-variational-99}. 
Before choosing the form of the approximate posterior, observe that the posterior distribution can be written as,
\begin{align}
p(\g, \h, \u, \v | \y) = p(\g | \u, \y) p(\h | \v, \y) p(\u, \v | \y).
\end{align}
Here we recall the modeling assumption that each set of the inducing variables is the \emph{sufficient statistics} of the corresponding latent process. 
This motivates the use of a distribution where the true posteriors over $\g$ and $\h$ are replaced by their conditional distributions given the inducing variables:
\begin{align}
q(\g, \h, \u, \v | \y)
= p(\g|\u) p(\h|\v) q(\u,\v),
\end{align}
with
\begin{align}
q(\u,\v) &= \prod_{j=1}^Q \underbrace{\Normal(\u_j; \mgj, \Sgj)}_{q(\u_j)} \prod_{i=1}^P  \underbrace{\Normal(\v_i; \mhi, \Shi)}_{q(\v_i)}.
\end{align}
This technique has been used in \citep{titsiasvariational,hensmangaussian} to derive variational inference for the single output case.
Since the conditionals $p(\g | \u)$ and $p(\h | \v)$ are given (Equation \ref{eq:gu} and \ref{eq:hv}), inference requires only learning of $q(\u, \v)$ that minimize the divergence between the approximate and true posteriors, which becomes small when the sufficiency condition holds with greater certainty.

\noindent
In variational inference, this is done by optimizing the evidence lower bound (ELBO) of the log marginal:
\begin{align}
%\nonumber
%\log p(\y) \ge& \int q(\u, \v) \log \frac{p(\y | \u, \v) p(\u, \v)}{q(\u, \v)} \der \u \der \v \\
%\nonumber
%=& \int q(\u, \v) \log p(\y | \u, \v)  \der \u \der \v 
%+ \int q(\u, \v) \log \frac{p(\u, \v)}{q(\u, \v)} \der \u \der \v \\
\nonumber
&\log p(\y) \ge \int q(\u, \v) \log p(\y | \u, \v)  \der \u \der \v \\
&- \sum_{j=1}^Q \KL[q(\u_j) || p(\u_j)] - \sum_{i=1}^P \KL[q(\v_i) || p(\v_i)],
\label{eq:elbo}
\end{align}
which is derived using Jensen's inequality and the fact that both of $q(\u, \v)$ and $p(\u, \v)$ fully factorize.
Since $q(\u_j), q(\v_i), p(\u_j), p(\v_i)$ are all multivariate Gaussian distributions, the KL divergences are analytically tractable and require $\calO(M^3)$ computation. To compute the expected likelihood term in the ELBO we first see that
\begin{align}
\nonumber
\log \text{ } &p(\y | \u, \v)
% &= \log \Eb{p(\y | \g, \h)}_{p(\g,\h | \u, \v)} \\
% \nonumber
\ge \Eb{\log p(\y | \g, \h)}_{p(\g,\h | \u, \v)}  \\
&= \sum_{i=1}^P \sum_{n=1}^N \Eb{\log p(y_{in} | \g_n, h_{in}) }_{p(\g_n | \u) p(\h_{in} | \v_i)} 
\end{align}
where $\g_n = \{g_{jn} = (\g_j)_n\}_{j=1}^Q$.
The inequality is due to Jensen's inequality and the equality is due to the fact that the likelihood fully factorizes.
\newcommand{\Ahi}{\A^h_i}
\newcommand{\Zi}{\Z_i}
The ELBO can be computed by first solving for the individual expectations 
$l_{in} \define \Eb{\log p(y_{in} | \g_n, h_{in}) }_{p(\g_n | \u) p(\h_in | \v_i)}$ (see the supplementary material for detailed derivation), which are then substituted into equation \ref{eq:elbo} to give:
\begin{align}
\nonumber
&\log p(\y)
\ge \sum_{i,n}
\bigg( \log  \Normal(y_{in}; \tilde{\mu}_{in}, \beta_i^{-1})
          - \frac{1}{2} \beta_i \sum_{j=1}^Q w_{ij}^2 \tilde{k}_{nn} \\ \nonumber
         &- \frac{1}{2} \beta_i \tilde{k}^h_{inn}
         - \frac{1}{2} \beta_i \trace  \sum_{j=1}^Q w_{ij}^2 \S_j \mat{\Lambda}_{jn} - \beta_i \frac{1}{2} \trace \S^h_i \mat{\Lambda}_{in} 
\bigg) \\
&- \sum_{j=1}^Q \KL[q(\u_j) || p(\u_j)] - \sum_{i=1}^P \KL[q(\v_i) || p(\v_i)]  \define \calL,
\end{align}
where, with the help of auxiliary matrices $\A_j = k(\X,\Zj)k(\Zj,\Zj)^{-1}$ and  $\Ahi = k(\X_i,\Zhi)k(\Zhi,\Zhi)^{-1}$, 
\begin{align}
\tilde{\mu}_{in}
%&= \sum_{j=1}^Q w_{ij} k(\x_n, \Zj)k(\Zj,\Zj)^{-1}\m_j + k(\x_n, \Zhi)k(\Zhi,\Zhi)^{-1}\mhi \\
&= \sum_{j=1}^Q w_{ij} \A_j(n,:) \m_j + \Ahi(n,:) \mhi \\
\mat{\Lambda}_{jn}
%&= k(\Zj,\Zj)^{-1} k(\Zj, \x_n) k(\x_n, \Zj) k(\Zj,\Zj)^{-1} \\
&= \A_j(n,:)^T \A_j(n,:)  \\
\mat{\Lambda}_{in}
%&= k(\Zhi,\Zhi)^{-1} k(\Zhi, \x_n) k(\x_n, \Zhi) k(\Zhi,\Zhi)^{-1}
&= \Ahi(n,:)^T \Ahi(n,:) 
\end{align}
where $\tilde{k}_{nn} = (\tilde{\K})_{nn}$, $\tilde{k}^h_{inn} = (\tilde{\K}^h_i)_{nn}$, $\mu_{jn} = (\Mu_j)_n$,  $\mu^h_{in} = (\Mu^h_i)_n$, and $\A_j(n,:)$ selects the $n$-th row vector of $\A_j$.

% More on this elbo term
\noindent Notice that this ELBO clearly generalizes the standard GP regression. In particular, setting $P = Q = 1$, $w_i = 1$ and $h_i(\x) = 0$ we recover the bound in Hensman et al \citet{hensmangaussian}.
Due to the decomposition of this bound, we can use stochastic gradient descent to learn the variational parameters.

\subsection{STOCHASTIC VARIATIONAL INFERENCE}
\newcommand{\oi}{\vec{o}_i}
% some notation

The optimal posteriors are found by setting the gradients of the lowerbound $\calL$ wrt to the parameters of $q(\u_j)$ and $q(\v_i)$.
Recall that in the model, different outputs can be observed at different inputs (i.e. the case of missing values).
Let $\oi$ be the indice of the observed inputs of the output dimension $i$.
We denote its set of observed inputs and targets as: $\X_i = \X(\oi,:)$ and $\y_i = y_{i}(\oi)$.

%\noindent As a function of the parameters of $q(\u_j)$, the lowerbound $\calL$ is:
%\begin{align}
%\nonumber
%\calL^g_j \define&
% \sum_{i=1}^P \log \Normal(\y_i; \sum_{j=1}^Q w_{ij} \A_j(\oi,:) \m_j + \Ahi \mhi, \beta_i^{-1} \I)  \\
% \nonumber
% &- \frac{1}{2} \sum_{i=1}^P \bigg(\beta_i \trace w_{ij}^2 \tilde{\K}_j(\oi,\oi) 
% + \beta_i \trace w_{ij}^2 \S_j \A_j(\oi,:)^T \A_j(\oi,:) \bigg)
% \\
% &- \frac{1}{2} \log |k(\Zj,\Zj) \S_j^{-1}| -\frac{1}{2} \trace k(\Zj,\Zj)^{-1} (\m_j \m_j^T + \S_j) ,
%\end{align}
%where $\A_j = k(\X,\Zj)k(\Zj,\Zj)^{-1}$, which gives $\A_j(\oi,:) = k(\X_i,\Zj) k(\Zj,\Zj)^{-1}$, and  
%$\Ahi = k(\X_i,\Zhi)k(\Zhi,\Zhi)^{-1}$. \\

%------------------------------------------
% derivatives of q(u_j)
\newcommand{\Lgj}{\calL^g_j}
\newcommand{\ynoj}{\y_i^{\backslash j}}
\newcommand{\Kjzz}{\mat{K}_{jzz}}
\noindent The derivatives of $\calL$ wrt $\m_j$ and $\S_j$ are given by:
\begin{align}
\nonumber
\deriv{\calL}{\m_j} 
=& \sum_{i=1}^P \beta_i w_{ij} \A_j(\oi,:)^T \ynoj \\
&- \bigg[\Kjzz^{-1} + \sum_{i=1}^P \beta_i w_{ij}^2 \A_j(\oi,:)^T \A_j(\oi,:) \bigg] \m_j \\
\deriv{\calL}{\S_j} 
=& \frac{1}{2} \S_j^{-1} - \frac{1}{2} \bigg[ \Kjzz^{-1} + \sum_{i=1}^P \beta_i w_{ij}^2 \A_j(\oi,:)^T \A_j(\oi,:) \bigg],
\end{align}
where $\Kjzz = k(\Zj,\Zj)$ and $\y_i^{\backslash j} = \y_i - \Ahi(\oi,:) \mhi - \sum_{j' \neq j} w_{ij'} \A_{j'}(\oi,:) \m_{j'}$.

%-------------------------------------------
%  derivatives of q(v_i)
%\noindent As a function of the parameters of $q(\v_i)$, the lower bound $\calL$ is:
%\begin{align}
%\nonumber
%\Lhi \define&
% \log \Normal(\y_i; \sum_{j=1}^Q w_{ij} \A_j(\oi,:) \m_j + \Ahi \mhi, \beta_i^{-1} \I)
% - \frac{1}{2} \beta_i \trace \tilde{\K}^h_i(\oi,\oi)
% - \frac{1}{2} \beta_i \trace \Shi (\Ahi)^T \Ahi
% \\
%  &- \frac{1}{2} \log |k(\Zhi,\Zhi) (\Shi)^{-1}| -\frac{1}{2} \trace k(\Zhi,\Zhi)^{-1} (\mhi (\mhi)^T + \Shi) ,
%\end{align}
\newcommand{\Lhi}{\calL^h_i}
\newcommand{\Kizz}{\mat{K}_{izz}}
\noindent The derivatives of $\calL$ wrt $\mhi$ and $\Shi$ are given by:
\newcommand{\ynoh}{\y_i^{\backslash h}}
\begin{align}
\nonumber
\deriv{\calL}{\mhi}
= & \beta_i (\Ahi(\oi,:))^T \ynoh \\ 
& - \bigg[\Kizz^{-1} +  \beta_i (\Ahi(\oi,:))^T \Ahi(\oi,:) \bigg] \m_i \\
\deriv{\calL}{\Shi} 
&= \frac{1}{2} \S_i^{-1} - \frac{1}{2} \bigg[ \Kizz^{-1} + \beta_i \Ahi(\oi,:))^T \Ahi(\oi,:) \bigg] ,
\end{align}
where $\ynoh = \y_i - \sum_{j=1}^Q w_{ij} \A_j(\oi,:) \m_j$ and $\Kizz = k(\Zhi, \Zhi)$.

% comment on computation
\noindent It can be seen that the derivatives of the parameters of $q(\v_i)$ only involve the observations of the output dimension $i$.
The derivatives of the parameters of $q(\u_j)$ involve the observations across all output dimensions but decompose as a sum of contributions from individual outputs.
Therefore, computation of the derivatives (and hence the update equations) can be distributed or parallelized easily.
This attractive property allows the model to scale to a very large number of inputs and outputs.

% Comment on other hyperparameters

\subsection{INDUCING INPUTS AND HYPER-PARAMETERS \label{sec:hyperparameters}}
%\subsubsection{Update Equations}
% and comment on the intuition of the update equations: e.g. y(x) - g(x) for h(x) 
\subsection{PREDICTION}
The predictive distribution of the $i$-th output for a test input $\x_*$ is 
\begin{align}
p(\fstar | \y, \x_*) = \int \Normal(\fstar; \sum_{j=1}^Q w_j g_{j*} + h_{i*}, 0) p(\g_* | \y, \x_*) p(h_{i*} | \y, \x_*) \der \g_* \der h_{i*},
\end{align}
where $p(\g_* | \y, \x_*) = \prod_{j=1}^Q \Normal(g_{j*}; \mu_{j*}, s_{j*})$ and $p(h_{i*} | \y, \x_*) = p(h_{i*}; \mu^h_{i*}, s^h_{i*})$ are the predictive distributions of the sparse GPs as given in eq. \ref{eq:sorprediction}.
Therefore we have:
\begin{align}
p(\fstar | \y, \x_*) = \Normal(\fstar; \sum_{j=1}^Q w_{ij} \mu_{j*} + \mu_{i*}, w_{ij}^2 s_{j*} + s_{i*}). 
\end{align}

%\begin{linenomath}
%\begin{align}
%\sum_{i,h} \lambda_i \lambda_h cov [f(\x, \x')]
%&= \sum_{i,h}   \sum_{j,j'} \lambda_i g(\x_i - \vs_j) k(\vs_j, \vs_{j'}) \lambda_h g(\x_h - \vs_j') \\
%\end{align}
%\end{linenomath}

