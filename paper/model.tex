Before diving into details of the specification, we discuss the philosophy and motivation behind our collaborative multioutput model.
To learn the outputs jointly, we need a mechanism through which information can be transferred among the outputs.
This is achieved in the model by allowing the outputs to share multiple sets of inducing variables.
For exposition, consider a set of inducing points for a moment.
Because they serve as the sufficient statistics of the individual outputs, a pattern common to the outputs can be encapsulated by these points when learning is done jointly. 
Different patterns in the data can be captured by having  multiple set of inducing variables.
It should be evident that these variables play a double pivotal role in the model: they represent the sufficient statistics of the sparse processes and they collaborate sharing of information across the outputs.

\newcommand{\Zj}{\Z_j}
\newcommand{\Zhi}{\Z^h_i}
Consider multioutput regression of $P$ tasks with inputs $\X = \{\x_n \in \calR^D\}_{n=1}^N$ and outputs $\y = \{\y_i\}_{i=1}^P$ where $\y_i = \{y_{in}\}_{n=1}^N$. 
Assume $Q$ latent functions with independent Gaussian process priors, $g_j(\x) \sim \GP(0, k_j(\cdot,\cdot))$, $j= 1 \hdots Q$.
We introduce the sets of \emph{inducing variables} $\u_j$, each of which comes from $g_j(\x)$, i.e., $\u_j$ contains the function values at the inducing inputs $\Z_j$, which live in the same space as $\X$.
These processes and inducing sets are shared by all of the outputs.
To accommodate greater flexibility, we further allow each output to have its own independent process $h_i(\x) \sim \GP(0, k^h_i(\cdot,\cdot))$ whose the sets of inducing inputs and variables are $\Z^h_i$ and $\v_i$ respectively, where $i = 1 \hdots P$.
% information is transfered via the inducing variables
We denote the collective variables: $\g = \{\g_j\}$, $\h = \{\h_i \}$, $\u = \{\u_j\}$, $\v = \{\v_i\}$, $\Z = \{\Zj\}$, and $\Z^h = \{\Zhi \}$ where $\g_j = \{g_j(\x_n)\}$, $\h_i = \{h_i(\x_n)\}$. 
The subscripts are $i = 1 \hdots P$, $j = 1 \hdots Q$, and $n = 1 \hdots N$.
For convenience, we shall assume the same number of inducing points, $M$, for all of the processes, however this is not imposed in practice.

Following the definition of the GPs and the independence of the processes, the \emph{prior} of the multioutput model can be written as:
\begin{align}
\label{eq:gu}
p(\g | \u) &= \prod_{j=1}^Q p(\g_j | \u_j) = \prod_{j=1}^Q \Normal(\g_j; \BigMu_j, \tilde{\K}_j )\\
p(\u) &= \prod_{j=1}^Q p(\u_j) = \prod_{j=1}^Q \Normal(\u_j; \vec{0}, k(\Zj, \Zj)) \\
\label{eq:hv}
p(\h | \v) &= \prod_{i=1}^P p(\h_i | \v_i) = \prod_{i=1}^P \Normal(\h_i; \BigMu^h_i, \tilde{\K}^h_i)\\
p(\v) &= \prod_{i=1}^P p(\v_i) = \prod_{i=1}^P \Normal(\v_i; \vec{0}, k(\Zhi, \Zhi)), \\
 \BigMu_j &= k(\X,\Zj)k(\Zj,\Zj)^{-1}\u_j \\
\BigMu^h_i &= k(\X,\Zhi)k(\Zhi,\Zhi)^{-1}\v_i \\
\tilde{\K}_j &= k_j(\X,\X) - k(\X,\Zj)k(\Zj,\Zj)^{-1}k(\Zj,\X) \\
\tilde{\K}^h_i &= k^h_i(\X,\X) - k(\X,\Zhi)k(\Zhi,\Zhi)^{-1}k(\Zhi,\X).
\end{align}
In the above equations and hereafter, we omit the subscripts $j,h,i$ from the kernels $k_j(\cdot,\cdot)$ and $k^h_i(\cdot,\cdot)$ when it is clear from the parameters inside the parentheses which covariance function is in action. 

The \emph{likelihood} with standard iid Gaussian likelihood is given by:
\begin{align}
p(\y | \g, \h ) = \prod_{i=1}^P \prod_{n=1}^N \Normal( y_{in} ; \sum_{j=1}^Q w_{ij} g_j(\x_n) + h_i(\x_n), \beta_i^{-1}),
\end{align}
which says that $y_{in}$ is a linear combination of the latent functions with weight $w_{ij}$ for the $j$th shared latent process plus a contribution from the individual process $h_i$.
As the latent values $\g$ are specified conditioned on the inducing variables $\u$, this construction implies that each output is a weighted combination of the inducing values.
We note that if $\u$ and $\v$ are marginalized out, we obtain the semiparametric latent factor models \citep{teh-et-al-aistats-05}.
However, doing so is against the purpose of this model which encourage sharing of outputs via the inducing variables.
Furthermore, as we shall see in the next section, explicit representation of these variables is instrumental to scalable inference of the model.

%\textbf{Augmented sparse GPs}
%Toward scalable modeling, we replace standard GPs with sparse GPs augmented with \textit{different} set of inducing inputs.
%This adds much flexibility to the model as each of the shared process $g_j(\x)$ can model a different pattern in the data with its own covariance function and inducing inputs. The roles of $g_j(\x)$ and $h_i(\x)$ can be quite different, so it is necessary that each has its own inducing inputs.
%Furthermore, $g_j(\x)$ can be seen as a \textit{global} function operating on the entire input space (of all output dimensions), while each $h_i(\x)$ operates only on the inputs of the $i$-th output, which can be a subspace of the input.
