% Sample LaTeX file for creating a paper in the Morgan Kaufmannn two
% column, 8 1/2 by 11 inch proceedings format.

\documentclass[]{article}
\usepackage{proceed2e}
\usepackage{hyperref}
\usepackage{url}
\usepackage{amsmath}
\usepackage{graphicx}
\usepackage{bm}
\usepackage{hyperref}
\usepackage{natbib}
\usepackage{latexsym,amsbsy,amssymb,color,xspace, booktabs}

% genereal purpose macros here
\newcommand{\marg}{\marginpar}
\newcommand{\todo}[1]{{\color{red}{TODO: #1}}}
\newcommand{\commentOut}[1]{} 
\newcommand{\add}[1]{\textcolor{red}{#1}} % To highligh modified text

% abbreviations
\newcommand{\etal}{et al.\xspace}
\newcommand{\ie}{i.e.\xspace}
\newcommand{\eg}{e.g.\xspace}
\newcommand{\Ie}{I.e.\xspace}



% Matrices and vectors 
\newcommand{\mat}[1]{\mathbf{#1}}
\renewcommand{\vec}[1]{ \mathbf{#1} } % math bold
\newcommand{\vecS}[1]{\boldsymbol{ #1 }  } % this for boldsymbols
\newcommand{\vecentry}[2]{{\mathrm #1_{#2}}}
\newcommand{\matentry}[3]{{\mathrm #1_{#2,#3}}}
\newcommand{\matcol}[2]{\mat{#1}_{\cdot #2}}
\newcommand{\matrow}[2]{\mat{#1}_{#2 \cdot}}

 % Matrices here
 \newcommand{\A}{\mat{A}}
\newcommand{\B}{\mat{B}}
\newcommand{\C}{\mat{C}}
\newcommand{\D}{\mat{D}}
\newcommand{\F}{\mat{F}}
\newcommand{\G}{\mat{G}}
\newcommand{\I}{\mat{I}}
\renewcommand{\L}{\mat{L}}
\newcommand{\X}{\mat{X}}
\newcommand{\W}{\mat{W}}
\newcommand{\Y}{\mat{Y}}
\newcommand{\Z}{\mat{Z}}
\renewcommand{\S}{\mat{S}}
\newcommand{\K}{\mat{K}}
\newcommand{\U}{\mat{U}}

% Vectors here
\newcommand{\f}{\vec{f}}
\newcommand{\g}{\vec{g}}
\newcommand{\h}{\vec{h}}
\newcommand{\m}{\vec{m}}
\renewcommand{\t}{\vec{t}}
\renewcommand{\u}{\vec{u}}
\newcommand{\w}{\vec{w}}
\newcommand{\x}{\vec{x}}
\newcommand{\y}{\vec{y}}
\newcommand{\z}{\vec{z}}

 % Vectorial greek letters
  \newcommand{\vecmu}{\vecS{\mu}}
 \newcommand{\vectheta}{\vecS{\theta}}
 \newcommand{\vecalpha}{\vecS{\alpha}}
 \newcommand{\vecphi}{\vecS{\phi}}
 \newcommand{\veceta}{\vecS{\eta}}
 \newcommand{\vecepsilon}{\vecS{\epsilon}} 
 
% caligraphic alphabet
\newcommand{\calA}{\mathcal{A}}
\newcommand{\calC}{\mathcal{C}}
\newcommand{\calD}{\mathcal{D}} 
\newcommand{\calL}{\mathcal{L}}
\newcommand{\calM}{\mathcal{M}}
\newcommand{\calT}{\mathcal{T}}
\newcommand{\calU}{\mathcal{U}}
\newcommand{\calX}{\mathcal{X}}
\newcommand{\calF}{\mathcal{F}}
\newcommand{\calE}{\mathcal{E}}
\newcommand{\calH}{\mathcal{H}}
\newcommand{\calI}{\mathcal{I}}
\newcommand{\calS}{\mathcal{S}}
\newcommand{\calY}{\mathcal{Y}}
\newcommand{\calO}{\mathcal{O}}
\newcommand{\calQ}{\mathcal{Q}}
\newcommand{\calR}{\mathcal{R}}
\newcommand{\calV}{\mathcal{V}}


% blackboard alphabet 
\newcommand{\setR}{\mathbb{R}}
\newcommand{\setE}{\mathbb{E}}
\newcommand{\setV}{\mathbb{V}}
\newcommand{\setI}{\mathbb{I}}
\newcommand{\setX}{\mathbb{X}}
\newcommand{\setT}{\mathbb{T}}



% Useful math operators
\newcommand{\Sum}{{\displaystyle \sum}}
\DeclareMathOperator{\vect}{vec}
\DeclareMathOperator{\diag}{diag}
\DeclareMathOperator*{\argmax}{argmax}                 
\DeclareMathOperator*{\argmin}{argmin}                 
\providecommand{\abs}[1]{\lvert#1\rvert}
\providecommand{\norm}[1]{\lVert#1\rVert}
\newcommand{\notp}[1]{\stackrel{\neg}{#1}} % symbol for not present


% matrix products
\newcommand{\kron}{\otimes}
\newcommand{\hada}{\bullet}

% statistics
\newcommand{\expectation}{\mathbb{E}}
\newcommand{\Eb}[1]{\left\langle #1 \right\rangle} % Expectation in angle brackets
\newcommand{\variance}{\mathbb{V}}
\newcommand{\Normal}{\mathcal{N}}  
\newcommand{\avg}[1]{\overline{#1}}
\newcommand{\KL}{\text{KL}}

% GP things
\newcommand{\GP}{\mathcal{GP}}
\newcommand{\kernel}{\kappa}

% Other maths
\newcommand{\deriv}[2]{\frac{\partial{#1}}{\partial{#2}}}
\newcommand{\trace}{\mbox{ \rm tr }}
\renewcommand{\det}[1]{\lvert#1\rvert}
\newcommand{\defeq}{\stackrel{\text{\tiny def}}{=}}
\newcommand{\idx}{\mathcal{I}}
\newcommand{\T}{\text{T}}
\newcommand{\mth}{\mathrm{th}} 


\newcommand{\bigO}{\calO}
\newcommand{\define}{\overset{\Delta}{=}}




















\newcommand{\gprnmf}{{\sc{gprn-mf}}}
\newcommand{\gprnnpv}{{\sc{gprn-npv}}}
\newcommand{\igp}{{\sc{igp}}}
\newcommand{\npv}[1]{{\sc{npv#1}}}
\newcommand{\fitc}{{\sc{fitc}}}

\newcommand{\vecu}{\vec{u}}
\newcommand{\vecf}{\vec{f}}
\newcommand{\vecy}{\vec{y}}
\newcommand{\vecw}{\vec{w}}
\newcommand{\vecfhat}{\hat{\vec{f}}}
\newcommand{\kernelf}{\kernel_{f}}
\newcommand{\kernelw}{\kernel_{w}}
\newcommand{\vecthetaf}{\vectheta_{f}}
\newcommand{\vecthetaw}{\vectheta_{w}}
\newcommand{\sigmaf}{\sigma_{f}}
\newcommand{\sigmay}{\sigma_{y}}
\newcommand{\Cu}{\mat{\C}_{u}}
\newcommand{\elbo}{\calL} % Evidence lower bound
\newcommand{\entropy}{\calH}
\newcommand{\p}{P} % Number of outputs
\newcommand{\q}{Q} % Number of latent functions
\newcommand{\vecfj}{\vec{f}_{j}}
\newcommand{\vecmufj}{\vecS{\mu}_{\text{f}_{\text{j}}}}
\newcommand{\vecmufk}{\vecS{\mu}_{\text{f}_{\text{k}}}}
\newcommand{\Sigmafj}{\mat{\Sigma}_{\text{f}_{\text{j}}}}
\newcommand{\vecwij}{\vec{w}_{ij}}
\newcommand{\vecmuwij}{\vecS{\mu}_{\text{w}_{\text{ij}}}}
\newcommand{\vecmuwik}{\vecS{\mu}_{\text{w}_{\text{ik}}}}
\newcommand{\Sigmawij}{\mat{\Sigma}_{\text{w}_{\text{ij}}}}
\newcommand{\matY}{\mat{Y}}
\newcommand{\vecmufn}{\vecS{\nu}_{\text{f}_{\text{n}}}} % These are for data structures for a fixed observation
\newcommand{\matMuwn}{\mat{\Omega}_{\text{w}_{\text{n}}}} % These are for data structures for a fixed observation
\newcommand{\Kf}{\mat{K}_{f}}
\newcommand{\Kw}{\mat{K}_{w}}
\newcommand{\Kfinv}{\mat{K}_{f}^{-1}}
\newcommand{\Kwinv}{\mat{K}_{w}^{-1}}

\newcommand{\vecyn}{\matcol{Y}{n}^{T}}
\newcommand{\sigmak}{\sigma_{k}}
\newcommand{\vecystar}{\vec{y}^*}
\newcommand{\vecxstar}{\vec{x}^*}
\newcommand{\vecWstar}{\vec{W}^*}
\newcommand{\vecfstar}{\vec{f}^*}
\newcommand{\der}{\mathrm{d}}
\newcommand{\vecmuw}{\vecS{\mu}_{\textbf{w}}}
\newcommand{\vecmuf}{\vecS{\mu}_{\textbf{f}}}
\newcommand{\Kwstar}{\Kw^*}
\newcommand{\Kfstar}{\Kf^*}

\newcommand{\sigmanoise}{\sigma_{\text{n}}}
\newcommand{\sigmoid}{\text{sig}}
\newcommand{\vf}{\vec{f}}
\newcommand{\vy}{\vec{y}}
\newcommand{\vu}{\vec{u}}
\newcommand{\vz}{\vec{z}}
\newcommand{\vx}{\vec{x}}
\newcommand{\vw}{\vec{w}}
\newcommand{\vg}{\vec{g}}
\newcommand{\vh}{\vec{h}}
\newcommand{\vm}{\vec{m}}
\newcommand{\matU}{\mat{U}}
\newcommand{\matX}{\mat{X}}
\newcommand{\Xu}{\mat{X}_u}
\newcommand{\bs}{\boldsymbol}
\newcommand{\xstar}{x_*}
\newcommand{\vxstar}{\vx_*}
\newcommand{\ystar}{y_*}
\newcommand{\zstar}{z_*}
\newcommand{\vfstar}{\vf_*}
\newcommand{\fstar}{f_*}
\newcommand{\Xuk}{\Xu^{(k)}}
\newcommand{\Kab}[2]{\mat{K}_{\vec{#1},\vec{#2}}}

% model specification
\newcommand{\BigMu}{\bs{\mu}}
\newcommand{\BigSigma}{\mat{\Sigma}}
\newcommand{\Mu}{\BigMu}
\newcommand{\Muk}{\bs{\mu}_k}
\newcommand{\Sigmak}{\mat{\Sigma}_k}
\newcommand{\hatMuk}{\bs{\hat \mu}_k}
\newcommand{\hatSigmak}{\mat{\hat \Sigma}_k}
\newcommand{\hatMukprime}{\bs{\hat \mu}_{k'}}
\newcommand{\hatSigmakprime}{\mat{\hat \Sigma}_{k'}}
\newcommand{\thetak}{\bs{\theta}_k}
\newcommand{\BigTheta}{\bs{\theta}}
\newcommand{\bsigma}{\bs{\sigma}}
\newcommand{\map}{\text{\sc{map}}}
\newcommand{\zmap}{\vz_\text{\sc{map}}}
\newcommand{\fuk}{\vec{g}_k}
\newcommand{\fu}{\vec{g}}
\newcommand{\kernelk}{k^{\BigTheta_k}}
\newcommand{\matUk}{\mat{U}_k}
\newcommand{\matXk}{\mat{X}_k}
\newcommand{\Lambdak}{\Lambda^{\BigTheta_k}}
\newcommand{\matK}{\mat{K}}
\newcommand{\matLambdak}{\mat{\Lambda}_k}

% variational inference
\newcommand{\Mufk}{\BigMu_k}
\newcommand{\Sigmafk}{\BigSigma_k}
\newcommand{\Kuk}{\matK^{(k)}_{\vu,\vu}}
\newcommand{\Psik}{\mat{\Psi}_k}
\newcommand{\Gammak}{\mat{\Gamma}_k}


\title{Collaborative Multi-output Gaussian Processes: \\ Supplementary Material}

\author{} % LEAVE BLANK FOR ORIGINAL SUBMISSION.
          % UAI  reviewing is double-blind.

% The author names and affiliations should appear only in the accepted paper.
%
%\author{ {\bf Harry Q.~Bovik\thanks{Footnote for author to give an
%alternate address.}} \\
%Computer Science Dept. \\
%Cranberry University\\
%Pittsburgh, PA 15213 \\
%\And
%{\bf Coauthor}  \\
%Affiliation          \\
%Address \\
%\And
%{\bf Coauthor}   \\
%Affiliation \\
%Address    \\
%(if needed)\\
%}

\begin{document}

\maketitle

\section{Gaussian Identities}
Let $\g = \{\g_j\}_{j=1}^q$, $\h$, and $\y$ be random variables with multivariate Gaussian distributions: 
$p(\y | \g, \h) = \Normal(\y; \sum_{j=1}^Q \W_j \g_j + \W \h, \beta^{-1} \I)$, $p(\g_j) = \Normal(\g_j; \m_j, \S_j)$, and $p(\h) = \Normal(\h; \m, \S)$.
The following identity is important in deriving the evidence lower bound:
\begin{align}
\nonumber
\int& \log p(\y | \g, \h) \prod_{j=1}^q p(\g_j) p(\h) \der \g \der \h \\
\nonumber
=& \log \Normal(\y; \sum_{j=1}^q \W_j \m_j + \W \m, \beta^{-1} \I) \\
& - \frac{1}{2} \beta \trace \W^T \W \S - \frac{1}{2} \beta \trace \sum_{j=1}^q \W_j^T \W_j \S_j.
 \label{eq:identity}
\end{align}
The identity can be proved by using this fact: 
\begin{align}
\nonumber
&\int (\W\x - \Mu)^T \BigSigma^{-1} (\W \x - \Mu) \Normal(\x; \m, \S) \der \x \\
&= (\Mu - \W\m)^T \BigSigma^{-1} (\Mu - \W\m) + \trace \W^T \BigSigma^{-1} \W \S.
\end{align}


\section{Derivation of the Variational Lower Bound}
The variational lower bound of the log marginal (eq. 13 in the main text) is given by:
\begin{align}
\nonumber
\log p(\y) \ge& \int q(\u, \v) \log \frac{p(\y | \u, \v) p(\u, \v)}{q(\u, \v)} \der \u \der \v \\
\nonumber
%=& \int q(\u, \v) \log p(\y | \u, \v)  \der \u \der \v  \\ 
%\nonumber
&+ \int q(\u, \v) \log \frac{p(\u, \v)}{q(\u, \v)} \der \u \der \v \\
\nonumber
=& \int q(\u, \v) \log p(\y | \u, \v)  \der \u \der \v \\
\nonumber
&- \sum_{j=1}^Q \KL[q(\u_j) || p(\u_j)] - \sum_{i=1}^P \KL[q(\v_i) || p(\v_i)],
\end{align}

Since the KL terms are analytically tractable, we compute the first term in the above sum. 
This is done first by deriving a lower bound to the likelihood $p(\y | \u, \v)$ (eq. 14 in the main text).
\begin{align}
\nonumber
\log \text{ } p(\y | \u, \v)
&= \log \Eb{p(\y | \g, \h)}_{p(\g,\h | \u, \v)} \\
\nonumber
&\ge \Eb{\log p(\y | \g, \h)}_{p(\g,\h | \u, \v)}  \\
\nonumber
&= \sum_{i=1}^P \sum_{n=1}^N \Eb{\log p(y_{in} | \g_n, h_{in}) }_{p(\g | \u) p(\h_i | \v_i)} 
\end{align}
Applying the identity in eq. \ref{eq:identity}, 
the expectation of an individual likelihood term with respect to the posterior distribution is given by: 
\begin{align}
\nonumber
l_{in} =& \int \log p(y_{in} | \g_n, h_{in}) \prod_{j=1}^Q p(g_{jn} | \u_j) p(h_{in} | \u_i) \der \g_n \der h_{in} \\
\nonumber
=& \log \Normal(y_{in}; \sum_{j=1}^Q w_{ij} \mu_{jn}+ \mu^h_{in}, \beta_i^{-1}) \\
&- \frac{1}{2} \beta_i \sum_{j=1}^Q w_{ij}^2 \tilde{k}_{jnn} 
- \frac{1}{2} \beta_i \tilde{k}^h_{inn},
\end{align}
where $\tilde{k}_{jnn} = (\tilde{\K}_j)_{nn}$, $\tilde{k}^h_{inn} = (\tilde{\K}^h_i)_{nn}$, $\mu_{jn} = (\Mu_j)_n$, and $\mu^h_{in} = (\Mu^h_i)_n$.

Substituting $l_{in}$ into the expression for the lower bound of $\log p(\y | \u, \v)$ (again, this is eq. 14 in the main text), and applying the identity in eq. \ref{eq:identity} to carry out the integral we obtain the lower bound as given in the main text.

% previous detailed derivation
%\begin{align}
%\nonumber
%l_{in} &= \int \log p(y_{in} | g_n, h_{in}) p(\g | \ug) p(\h_i | \u_i) \der \g \der \h_i \\
%\nonumber
%&= \int \log \Normal(y_{in} ; w_i g_n + h_{in}, \beta_i^{-1}) 
%\Normal(g_n ; \mu_{gn}, \tilde{k}_{gnn})
%\Normal(h_{in} ; \mu_{in}, (\tilde{k}_{inn}) \der g_n \der h_{in} \\
%\nonumber
%&= -\frac{1}{2} \log 2 \pi \beta_i^{-1} - \frac{1}{2} \int (y_{in} - w_i g_n - h_{in}) \beta_i (y_{in} - w_i g_n - h_{in})
%\Normal(g_n ; \mu_{gn}, \tilde{k}_{gnn})
%\Normal(h_{in} ; \mu_{in}, \tilde{k}_{inn}) \der g_n \der h_{in} \\
%\nonumber
%&= -\frac{1}{2} \log 2 \pi \beta_i^{-1} - \frac{1}{2} \int \left[(y_{in} - h_{in} - w_i \mu_{gn}) \beta_i (y_{in} - h_{in} - w_i \mu_{gn}) + w_i^2 \beta_i \tilde{k}_{gnn} \right] 
%\Normal(h_{in} ; \mu_{in}, \tilde{k}_{inn}) \der h_{in} \\
%\nonumber
%&= -\frac{1}{2} \log 2 \pi \beta_i^{-1} - \frac{1}{2} w_i^2 \beta_i \tilde{k}_{gnn}
%- \frac{1}{2} \beta_i \tilde{k}_{inn} - \frac{1}{2} (y_{in} - w_i \mu_{gn} - \mu_{in}) \beta_i (y_{in} - w_i \mu_{gn} - \mu_{in}) \\
%&= \log \Normal(y_{in}; w_i \mu_{gn} + \mu_{in}, \beta_i^{-1})  - \frac{1}{2} \beta_i (w_i^2 \tilde{k}_{gnn} + \tilde{k}_{inn}),
%\end{align}

\section{Derivatives of the ELBO}
For exposition, we derive the gradients of the lower bound for the case of a single GP (i.e. the bound in \citet{hensmangaussian}).
The derivatives of the ELBO of the collaborative multioutput GPs model are typically linear combination of the derivatives here.
The lower bound as a function of all parameters is
\begin{align}
\nonumber
\calL
=& \log \Normal(\y; \K_{NM} \K_{MM}^{-1} \m, \beta^{-1}\I) \\
\nonumber
 &- \frac{1}{2} \beta \trace \tilde{\K}
 - \frac{1}{2} \beta \trace (\S\K_{MM}^{-1} \K_{MN} \K_{NM} \K_{MM}^{-1}) \\  \nonumber
&- \frac{1}{2} \left( \log |\K_{MM}| + \trace(\K_{MM}^{-1}(\m \m^T + \S)) \right) \\ \nonumber
=& \underbrace{\log \Normal(\y; \A \m, \beta^{-1}\I)}_{\calL_1}
 - \underbrace{\frac{1}{2} \beta \trace (\K_{NN} - \A\K_{MN})}_{\calL_2} \\
 \nonumber
&- \underbrace{\frac{1}{2} \left( \log |\K_{MM}| + \trace(\K_{MM}^{-1}(\m \m^T + \S)) \right)}_{\calL_4}\\
 & - \underbrace{\frac{1}{2} \beta \trace (\S\A^T\A)}_{\calL_3},
\end{align}
where $\A = \K_{NM} \K_{MM}^{-1}$.
Here $\K_{NM} = k(\X,\Z)$ is the cross-covariance matrix between the observed inputs and the inducing inputs and $\K_{MM} = k(\Z,\Z)$ is the auto-covariance matrix between the inducing inputs.
Notice that we have re-written the sum of individual terms in matrix form which will make the derivation easier and also the computation faster via vectorization.

\subsection{Derivative of the Noise Hyperparameter}
The derivative of the noise hyperparameter $\beta$ is easily computed as:
\begin{align}
\deriv{\calL}{\beta} = \frac{N}{2\beta} - \frac{1}{2} (\y - \A\m)^T (\y - \A\m) - \frac{\calL_2}{\beta} - \frac{\calL_3}{\beta}.
\end{align}
\subsection{Derivatives of the Covariance Hyperparameters}  
To simplify the math, we utilize the matrix $\A$ defined above.
Firstly, the derivative of $\A$ wrt a covariance hyperparameter $t$ is given by:
\begin{align}
\deriv{\A}{t} = \left(\deriv{\K_{NM}}{t} - \A \deriv{\K_{MM}}{t}\right)\K_{MM}^{-1}.
\end{align}
The derivatives of $\calL_1, \calL_2, \calL_3 \text{ and } \calL_4$ are thus given by:
\begin{align}
\deriv{\calL_1}{t} =& \beta (\y - \A\m)^T \deriv{\A}{t} \m \\
\deriv{\calL_2}{t} =& \frac{1}{2}\beta \trace \left(\deriv{\K_{NN}}{t} - \A \deriv{\K_{MN}}{t} - \deriv{\A}{t} \K_{MN}\right)
\end{align}
\begin{align}
\deriv{\calL_3}{t} =& \beta \trace \left(\A \S \deriv{\A^T}{t} \right) \\
\nonumber
\deriv{\calL_4}{t} =& \frac{1}{2}  \trace \left(\K_{MM}^{-1} \deriv{ \K_{MM}}{t}\right) \\
&- \frac{1}{2} \trace \left(\K_{MM}^{-1} \deriv{\K_{MM}}{t} \K_{MM}^{-1} (\m \m^T + \S) \right) 
\end{align}
The derivatives are then computed by taking the derivatives of the covariance matrices $\K_{NN} (\text{the diagonal only}), \K_{NM} \text{ and }  \K_{MM}$, hence the covariance function, wrt the hyperparameters. 

\subsection{Derivatives of the Inducing Inputs}
To compute the derivatives of $\calL$ wrt the inducing inputs, first notice that $\Z = \{\z_m\}_{m=1}^M$ can also be viewed as parameters of the covariance matrices $\K_{NM}$ and $\K_{MM}$.
Hence the derivative wrt a single dimension of an inducing input, i.e. $z_{mj}$, is the same as that of $\deriv{ \calL}{t}$.

%Since $MD$ parameters are needed for the inducing inputs, it appears that the derivatives of all inducing inputs would require $\calO(MD \times M^3)$ in computation.
%However, this complexity can actually be reduced to $\calO(DM^3)$ using the following lemma. \\
%
%\noindent \textbf{Lemma} Let $A, B$ be two matrices of size $N \times M$ and $M \times N$, respectively. Furthermore, $B$ has the property that only one of its rows or columns is non-zero. The complexity of $\trace(AB)$ is only $\calO(N)$. \\
%
%\noindent \textbf{Proof} Let $m <= M$ be the non-zero row of $B$. We have
%\begin{align}
%\trace (AB) = \sum_{i=1}^N \sum_{j=1}^M A_{ij} B_{ji} = \sum_{i=1}^N A_{im} B_{mi},
%\end{align}
%which clearly takes $\calO(N)$. It is easy to see that the lemma also holds when $B$ is symmetric and only one of its row and the corresponding column is non-zero. \\
%
%\noindent To exploit the property in the Lemma, we re-write $\frac{d \calL_1}{dt}, \frac{d \calL_2}{dt}, \frac{d \calL_3}{dt}, \frac{d \calL_4}{dt}$ by expanding $\frac{d\A}{dt}$ (here $t = z_{mj}$):

\noindent We re-write $\deriv{\calL_1}{t}, \deriv{\calL_2}{t}, \deriv{ \calL_3}{t}, \deriv{\calL_4}{t}$ by expanding $\deriv{\A}{t}$ (here $t = z_{mj}$):
% dL1
\begin{align}
\nonumber
\deriv{\calL_1}{t}
=& \beta \trace (\y - \A\m)^T \left(\deriv{\K_{NM}}{t} -  \A \deriv{ \K_{MM}}{t}\right)\K_{MM}^{-1} \m \\
\nonumber
=& \beta \trace \K_{MM}^{-1} \m (\y - \A\m)^T \deriv{\K_{NM}}{t} \\
&-\beta \trace \K_{MM}^{-1} \m (\y - \A\m)^T \A \deriv{\K_{MM}}{t}
\end{align}
%dL2
\begin{align}
\deriv{\calL_2}{t}
&= - \beta \trace \A^T \deriv{\K_{NM}}{t}
 + \frac{1}{2} \beta \trace \A^T \A \deriv{\K_{MM}}{t}
 \end{align}
% dL3
 \begin{align}
\deriv{\calL_3}{t}
&= \beta \trace \K_{MM}^{-1} \S \A^T \deriv{\K_{NM}}{t}
 - \beta \trace \K_{MM}^{-1} \S \A^T \A \deriv{\K_{MM}}{t}
 \end{align}
 % dL4
 \begin{align}
\deriv{\calL_4}{t}
 =& \frac{1}{2}  \trace \K_{MM}^{-1} \deriv{\K_{MM}}{t} \\
 \nonumber
& - \frac{1}{2} \trace \K_{MM}^{-1} (\m \m^T + \S) \K_{MM}^{-1} \deriv{\K_{MM}}{t} 
\end{align}
From the above 4 equations we get,
\begin{align}
\deriv{\calL}{t} = \trace \D_1 \deriv{\K_{NM}}{t} + \trace \D_2 \deriv{ \K_{MM}}{t},
\end{align}
where 
\begin{align}
\D_1 =& \beta \K_{MM}^{-1} \m (\y - \A\m)^T
 + \beta \A^T
 - \beta \K_{MM}^{-1} \S \A^T \\ \nonumber
\D_2 =& -\beta \trace \K_{MM}^{-1} \m (\y - \A\m)^T \A
 - \frac{1}{2} \beta \A^T \A -\frac{1}{2} \K_{MM}^{-1} \\
  &+ \beta \K_{MM}^{-1} \S \A^T \A	 
   + \frac{1}{2} \K_{MM}^{-1} (\m \m^T + \S) \K_{MM}^{-1}
\end{align}
%TODO: give the correct cost
Notice that $\D_1$ and $\D_2$ can be pre-computed with a cost of $\calO(M^3)$ (or $\calO(N_bM^2)$ if the minibatch size $N_b > M$).
The computational cost of taking derivatives of $MD$ inducing parameters is thus $\calO(M^3 + MDM) = \calO(M^3)$ as the cost of the two trace operators is $\calO(M)$ due to the fact that only $\calO(M)$ elements of $\deriv{\K_{MM}}{t}$ or $\deriv{\K_{NM}}{t}$ are non-zero.

For implementation with MATLAB, a loop over $M \times D$ parameters of the inducing inputs can be very slow for even moderate values of $M$ and $D$.
The aforementioned fact about $\deriv{\K_{MM}}{t}$ and $\deriv{\K_{NM}}{t}$ can be used to perform vectorized operations that compute the derivatives of all $M$ parameters given a specific dimension at a cost of $\calO(M^2)$.
The loop is the executed over the input dimension $D$, leading to a complexity of still $\calO(DM^2)$ only.

\bibliographystyle{plainnat}
\bibliography{references}

\end{document}
